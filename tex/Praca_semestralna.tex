\documentclass[a4paper, 12pt]{article}

\usepackage[T1]{fontenc}
\usepackage[polish]{babel} 
\usepackage[utf8]{inputenc} 
\let\lll\undefined
\usepackage{setspace}
\usepackage{fancyhdr}
\usepackage{hyperref}
\usepackage{pdfpages}
\usepackage{listings}
\usepackage{color}
\usepackage{graphicx}
\usepackage{enumitem}
\usepackage{latexsym}
\pagestyle{fancy} 
\hypersetup{
    colorlinks=true,
    linkcolor=blue,
    filecolor=magenta,      
    urlcolor=cyan,
}
\usepackage{geometry}
\newgeometry{tmargin=2.5cm, bmargin=2.5cm, lmargin=2.5cm, rmargin=2.5cm} 
\newcommand{\mainmatter}{\clearpage \cfoot{\thepage\ of \pageref{LastPage}}
\pagenumbering{arabic}} 
\lstset{language=bash}  
\begin{document}

	\begin{titlepage}
\includegraphics[width = 40mm]{logo.jpg}
		\begin{center}
    			\vspace{3cm}
    					\Large\textit{\textbf{Procesy biznesowe (definicje, kryteria i klasyfikacje, przykłady)}}
   			\vspace{4cm}
		\end{center} 

		\hfill\begin{minipage}{0.54\textwidth}
			\Large Wykonały:\newline
				1. Ivan Prakapets 295139 \newline
				2. Aliaksandr Karolik 295138
		\vspace{\baselineskip}
		\end{minipage}
		
		\hfill\begin{minipage}{0.54\textwidth}
			\Large Sprawdzający:\newline
		 		prof.dr hab.inż.Andrzej Dzieliński
\vspace{\baselineskip}
		\end{minipage}

		\hfill\begin{minipage}{0.54\textwidth}
			\Large Data:10.11.2019
			\vspace{\baselineskip}
		\end{minipage}
	\end{titlepage}
\newpage
\mainmatter
\setlength{\headheight}{15pt}
\doublespacing
\tableofcontents
\newpage

\linespread{0.5}
\setlist{nolistsep}

\section{Wstęp}

\section{Definicja procesu biznesowego}
\hspace*{1.5 cm}Poniżej zostaną zaprezentowane różne definicje pojęcia proces biznezosy:

\hspace*{1.5 cm}Proces biznesowy jest zbiorem czynności, ma jeden lub więcej rodzajów wejść i tworzy wartość wyjściową dla klienta. Proces biznesowy posiada swój cel, a oddziałują na niego zdarzenia zachodzące w świecie zewnętrznym lub w innych procesach.
(Hammer and Champy 1993)

\hspace*{1.5 cm} Proces biznesowy jest to zbiór powiązanych procedur lub działań, które wspólnie zapewniają osiągnięcie celu biznesowego lub celu polityki, zwykle w ramach struktury organizacyjnej definiującej funkcjonalność ról i zależności pomiędzy nimi. (Workflow Management Coalition 1999) 

\hspace*{1.5 cm}Proces biznesowy może być postrzegany jako struktura czynności zaprojektowanych na działania nakierowane na klienta końcowego i dynamiczne zarządzanie przepływami związanymi z produktami, informacją, środkami finansowymi, wiedzą i wizją.(Stock and Lambert 2001)        

\hspace*{1.5 cm}Proces biznesowy lub inaczej metoda biznesowa, która funkcjonuje w każdym przedsiębiorstwie. Są to zadania ze sobą powiązane, które prowadzą do osiągnięcia wyznaczonego efektu. Najważniejszym celem tego procesu jest zrozumieć klienta, dostawców i słabe strony tego procesu.(A. Bitkowska 2009, s.26)

\hspace*{1.5 cm}Określony zbiór czynności biznesowych, które stanowią niezbędne kroki w celu osiągnięcia celu biznesowego. Obejmuje on przepływ i użycie informacji oraz zasobów.(Object Management Group – BPMN v. 2.0 2011) 
\subsection{Podsumowanie czym jest proces biznesowy} 
\hspace{1.5 cm} Bazując na powyższych definicjach, można przyjąć, że proces biznesowy, to zbiór powiązanych ze sobą czynności, które przekształcają wejścia w wyjścia według określonych reguł, w oparciu o określone zasoby i w efekcie prowadzą do dostarczenia klientowi produktu/usługi realizując tym samym cele biznesowe organizacji.
\section{Kryteria i klasyfikacja procesów biznesowych}
\subsection{Rodzaje kryteriów}
\subsection{Rodzaje klasyfikacji}
\subsection{Róznica kryteria a klasyfikacji}
\subsection{Podsumowanie}

\section{Przykłady biznes procesów w różnych rodzajach firm}
\subsection{Prykład biznes procesu w firmie IT}
\subsubsection{Zakupienie nowych serwerów firmą jako proces biznesowy}
\subsubsection{Ogólny opis procesu}
\subsubsection{Problem przestarzełęgo sprzętu} 
\subsubsection{Problem mocy obliczeniowej} 
\subsubsection{Opis przepływu procesu biznesowego}

\subsection{Przykład biznes procesu w firmie marketingowej}
\subsubsection{Ogólny opis procesu}
\subsubsection{Problemy które proces rozwiązuje}
\subsubsection{Opis przepływu procesu biznesowego}

\section{Wnioski} 


\label{LastPage}~
\label{LastPageOfBackMatter}~		
\end{document}