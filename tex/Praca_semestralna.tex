\documentclass[a4paper, 12pt]{article}

\usepackage[T1]{fontenc}
\usepackage[polish]{babel} 
\usepackage[utf8]{inputenc} 
\let\lll\undefined
\usepackage{setspace}
\usepackage{fancyhdr}
\usepackage{hyperref}
\usepackage{pdfpages}
\usepackage{listings}
\usepackage{color}
\usepackage{graphicx}
\usepackage{enumitem}
\usepackage{latexsym}
\pagestyle{fancy} 
\hypersetup{
    colorlinks=true,
    linkcolor=blue,
    filecolor=magenta,      
    urlcolor=cyan,
}
\usepackage{geometry}
\newgeometry{tmargin=2.5cm, bmargin=2.5cm, lmargin=2.5cm, rmargin=2.5cm} 
\newcommand{\mainmatter}{\clearpage \cfoot{\thepage\ of \pageref{LastPage}}
\pagenumbering{arabic}} 
\lstset{language=bash}  
\begin{document}

	\begin{titlepage}
\includegraphics[width = 40mm]{logo.jpg}
		\begin{center}
    			\vspace{3cm}
    					\Large\textit{\textbf{Procesy biznesowe (definicje, kryteria i klasyfikacje, przykłady)}}
   			\vspace{4cm}
		\end{center} 

		\hfill\begin{minipage}{0.54\textwidth}
			\Large Wykonali:\newline
				1. Ivan Prakapets 295139 \newline
				2. Aliaksandr Karolik 295138
		\vspace{\baselineskip}
		\end{minipage}
		
		\hfill\begin{minipage}{0.54\textwidth}
			\Large Sprawdzający:\newline
		 		prof.dr hab.inż.Andrzej Dzieliński
\vspace{\baselineskip}
		\end{minipage}

		\hfill\begin{minipage}{0.54\textwidth}
			\Large Data:10.11.2019
			\vspace{\baselineskip}
		\end{minipage}
	\end{titlepage}
\newpage
\mainmatter
\setlength{\headheight}{15pt}
\doublespacing
\tableofcontents
\newpage

\linespread{0.5}
\setlist{nolistsep}

\section{Wstęp}
\hspace*{1.5 cm}Poniższy dokument jest  pracą semestralną na temat "Procesy biznesowe (definicje, kryteria i klasyfikacje, przykłady)". W tym dokumencie postaramy się zdefiniować co to jest proces biznesowy oraz zbadać jakie są kryteria i klasyfikacji procesów biznesowych. Również postaramy się rozpatrzyć kilka procesów biznesowych w różnych firmach aby na ich podstawie zaobserwować różne rodzaje procesów biznesowych. 
\section{Definicja procesu biznesowego}
\hspace*{1.5 cm}Poniżej zostaną zaprezentowane różne definicje pojęcia proces biznezosy:

\hspace*{1.5 cm}Proces biznesowy jest zbiorem czynności, ma jeden lub więcej rodzajów wejść i tworzy wartość wyjściową dla klienta. Proces biznesowy posiada swój cel, a oddziałują na niego zdarzenia zachodzące w świecie zewnętrznym lub w innych procesach.
(Hammer and Champy 1993)

\hspace*{1.5 cm} Proces biznesowy jest to zbiór powiązanych procedur lub działań, które wspólnie zapewniają osiągnięcie celu biznesowego lub celu polityki, zwykle w ramach struktury organizacyjnej definiującej funkcjonalność ról i zależności pomiędzy nimi. (Workflow Management Coalition 1999) 

\hspace*{1.5 cm}Proces biznesowy może być postrzegany jako struktura czynności zaprojektowanych na działania nakierowane na klienta końcowego i dynamiczne zarządzanie przepływami związanymi z produktami, informacją, środkami finansowymi, wiedzą i wizją.(Stock and Lambert 2001)        

\hspace*{1.5 cm}Proces biznesowy lub inaczej metoda biznesowa, która funkcjonuje w każdym przedsiębiorstwie. Są to zadania ze sobą powiązane, które prowadzą do osiągnięcia wyznaczonego efektu. Najważniejszym celem tego procesu jest zrozumieć klienta, dostawców i słabe strony tego procesu.(A. Bitkowska 2009, s.26)

\hspace*{1.5 cm}Określony zbiór czynności biznesowych, które stanowią niezbędne kroki w celu osiągnięcia celu biznesowego. Obejmuje on przepływ i użycie informacji oraz zasobów.(Object Management Group – BPMN v. 2.0 2011) 
\subsection{Podsumowanie czym jest proces biznesowy} 
\hspace{1.5 cm} Bazując na powyższych definicjach, można przyjąć, że proces biznesowy, to zbiór powiązanych ze sobą czynności, które przekształcają wejścia w wyjścia według określonych reguł, w oparciu o określone zasoby i w efekcie prowadzą do dostarczenia klientowi produktu/usługi realizując tym samym cele biznesowe organizacji.
\section{Kryteria i klasyfikacji procesów biznesowych}
\subsection{Rodzaje kryteriów}
\subsection{Rodzaje klasyfikacji}
\subsection{Róznica kryteria a klasyfikacji}
\subsection{Podsumowanie}

\section{Przykłady biznes procesów w różnych rodzajach firm}
\subsection{Prykład biznes procesu w firmie IT}
\subsubsection{Opis ogólny procesu}
\hspace*{1cm} Potrzebujemy zakupić nowe serwery z potrzeby zaspokojenia bieżącego zapotrzebowania na moc obliczeniową, przyrostu danych oraz wymiana obecnie posiadanego systemu na nowszy, bo jest brak kompatybilności z nowymi systemami operacyjnymi, brak wsparcia producenta, z tego wynika wysokie ryzyko awarii. Poniżej jest dokładne opisane problemy, informacje, wykresy oraz charakterystyki dotyczące obecnych i nowych serwerów.\newline
\subsubsection{Występujące problemy w firmie} 
		\begin{itemize}
		        \item \texttt{Na obecnym serwerze baza danych wzrasta się około 100 Gb co rok. Ze względu na duże przyrosty danych na obecnych serwerach brakuje miejsca. Brak przestrzeni na dane powoduje coraz wolniejsze działanie serwera. W przypadku awarii niemożliwym staje się szybka naprawa. W obecnej postój pracy wynosi 5 dni.}
	        	\item \texttt{Problemy z komunikacją oraz transferem danych między serwerami spowodowana przestarzałością sprzętu}
	        	\item \texttt{Niski poziom Uptime, poziom dostępności systemu określony w procentach. Informujący o czasie ciągłego i bezawaryjnego działania serwerów w ciągu roku. Można spodziewać się awarii w niedługim okresie.}
	        	\item \texttt{Okres gwarancji na sprzęt oraz oprogramowanie wygasł, wiąże się to z częstrzą możliwością występowania corac cięższych awarii.}
	        	\item \texttt{Przestarzały sprzęt wycofany już z produkcji i powszechnego użytku. Posiadany sprzęt zurzywa się i traci pierwotą niezawodność. Generuje to ryzyko braku możliwości wymiany części i podzespołów w przypadku uszkodzeń i awarii.}
	        	\item \texttt{ Stare oprogramowanie traci swoją kompatybilność z nowymi systemami i ma słabsze możliwości względem nowych.}
	        	\item \texttt{Niewystarczająca moc obliczeniowa powoduje powolną pracę serwerów. Dane przesyłane do aplikacji są coraz wolniej selekcjonowane oraz generowanie z długim czasem oczekiwania na wyniki. Wpływ na to ma szybkość procesowa, pamięć RAM, pojemność dysku, przepustowość, itp.}
		    \end{itemize}
	  \subsubsection{Wnioski} 
	
			 	\hspace*{1cm} Pozostaje mało miejsca i istnieje potrzeba rozszerzenia przestrzeni roboczej poprzez podłączenie dodatkowych urządzeń pamięci masowej lub zwiększenie pojemności istniejącej macierzy dyskowej. Niestety uniemożliwia to przestarzały sprzęt niekompatybilny z odpowiednikami dostępnymi na rynku.\newline
			\hspace*{1cm}  Wszystkie powyższe problemu powodują przerwy, awarie oraz  przestoje - \textbf{generuje to straty materialne oraz oszczerbki na reputacji firmy.}
			
			\newpage
			\subsubsection{Rozwiązanie problemów}
				TUTAJ COS JESZCZE
				\subsubsection{Klastry SQL Server 2019} 	
	\hspace*{1cm} Klastry to technologia, która pozwala łączyć kilka systemów komputerowych w całość. 
Cel klastra to z duplikowanie  serwerów i zapewnienie nieprzerwanego funkcjonowania aplikacji. W serwerach bazodanowych wszystkie informacje są zapamiętywane na każdej równolegle pracującej maszynie. W razie awarii systemy i urządzenia łączą się działającym serwerem, zawierającym te same dane.
\newline
	\hspace*{1cm}Klastry SQL Server 2019 to nowy sposób wykorzystania SQL Server do tworzenia wartościowych baz relacyjnych i przechowywania dużych zbiorów danych na jednej, skalowalnej platformie. Dzięki temu analiza danych oraz aplikacje łączące się z bazą są bardziej elastyczne i działają wydajniej. Duże klastry danych SQL Server 2019 zapewniają kompletną platformę i pomagają zwiększyć sukces  organizacji.
\subsubsection{Opis przepływu procesu biznesowego}

\subsection{Przykład biznes procesu w firmie marketingowej}
\subsubsection{Ogólny opis procesu}
\subsubsection{Problemy które proces rozwiązuje}
\subsubsection{Opis przepływu procesu biznesowego}

\section{Wnioski} 


\label{LastPage}~
\label{LastPageOfBackMatter}~		
\end{document}