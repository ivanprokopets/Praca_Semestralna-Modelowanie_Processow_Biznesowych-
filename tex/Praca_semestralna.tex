\documentclass[a4paper, 12pt]{article}

\usepackage[T1]{fontenc}
\usepackage[polish]{babel} 
\usepackage[utf8]{inputenc} 
\let\lll\undefined
\usepackage{setspace}
\usepackage{fancyhdr}
\usepackage{hyperref}
\usepackage{pdfpages}
\usepackage{listings}
\usepackage{color}
\usepackage{graphicx}
\usepackage{enumitem}
\usepackage{latexsym}
\pagestyle{fancy} 
\hypersetup{
    colorlinks=true,
    linkcolor=blue,
    filecolor=magenta,      
    urlcolor=cyan,
}
\usepackage{geometry}
\newgeometry{tmargin=2.5cm, bmargin=2.5cm, lmargin=2.5cm, rmargin=2.5cm} 
\newcommand{\mainmatter}{\clearpage \cfoot{\thepage\ of \pageref{LastPage}}
\pagenumbering{arabic}} 
\lstset{language=bash}  
\begin{document}

	\begin{titlepage}
\includegraphics[width = 40mm]{logo.jpg}
		\begin{center}
    			\vspace{3cm}
    					\Large\textit{\textbf{Procesy biznesowe (definicje, kryteria i klasyfikacje, przykłady)}}
   			\vspace{4cm}
		\end{center} 

		\hfill\begin{minipage}{0.54\textwidth}
			\Large Wykonali:\newline
				1. Ivan Prakapets 295139 \newline
				2. Aliaksandr Karolik 295138
		\vspace{\baselineskip}
		\end{minipage}
		
		\hfill\begin{minipage}{0.54\textwidth}
			\Large Sprawdzający:\newline
		 		prof.dr hab.inż.Andrzej Dzieliński
\vspace{\baselineskip}
		\end{minipage}

		\hfill\begin{minipage}{0.54\textwidth}
			\Large Data:10.11.2019
			\vspace{\baselineskip}
		\end{minipage}
	\end{titlepage}
\newpage
\mainmatter
\setlength{\headheight}{15pt}
\doublespacing
\tableofcontents
\newpage

\linespread{0.5}
\setlist{nolistsep}

\section{Wstęp}
\hspace*{1.5 cm}Poniższy dokument jest  pracą semestralną na temat "Procesy biznesowe (definicje, kryteria i klasyfikacje, przykłady)". W tym dokumencie postaramy się zdefiniować co to jest proces biznesowy oraz zbadać jakie są kryteria i klasyfikacji procesów biznesowych. Również postaramy się rozpatrzyć kilka procesów biznesowych w różnych firmach aby na ich podstawie zaobserwować różne rodzaje procesów biznesowych. 
\section{Definicja procesu biznesowego}
\hspace*{1.5 cm}Poniżej zostaną zaprezentowane różne definicje pojęcia proces biznezosy:

\hspace*{1.5 cm}Proces biznesowy jest zbiorem czynności, ma jeden lub więcej rodzajów wejść i tworzy wartość wyjściową dla klienta. Proces biznesowy posiada swój cel, a oddziałują na niego zdarzenia zachodzące w świecie zewnętrznym lub w innych procesach.
(Hammer and Champy 1993)

\hspace*{1.5 cm} Proces biznesowy jest to zbiór powiązanych procedur lub działań, które wspólnie zapewniają osiągnięcie celu biznesowego lub celu polityki, zwykle w ramach struktury organizacyjnej definiującej funkcjonalność ról i zależności pomiędzy nimi. (Workflow Management Coalition 1999) 

\hspace*{1.5 cm}Proces biznesowy może być postrzegany jako struktura czynności zaprojektowanych na działania nakierowane na klienta końcowego i dynamiczne zarządzanie przepływami związanymi z produktami, informacją, środkami finansowymi, wiedzą i wizją.(Stock and Lambert 2001)        

\hspace*{1.5 cm}Proces biznesowy lub inaczej metoda biznesowa, która funkcjonuje w każdym przedsiębiorstwie. Są to zadania ze sobą powiązane, które prowadzą do osiągnięcia wyznaczonego efektu. Najważniejszym celem tego procesu jest zrozumieć klienta, dostawców i słabe strony tego procesu.(A. Bitkowska 2009, s.26)

\hspace*{1.5 cm}Określony zbiór czynności biznesowych, które stanowią niezbędne kroki w celu osiągnięcia celu biznesowego. Obejmuje on przepływ i użycie informacji oraz zasobów.(Object Management Group – BPMN v. 2.0 2011) 
\subsection{Podsumowanie czym jest proces biznesowy} 
\hspace{1.5 cm} Bazując na powyższych definicjach, można przyjąć, że proces biznesowy, to zbiór powiązanych ze sobą czynności, które przekształcają wejścia w wyjścia według określonych reguł, w oparciu o określone zasoby i w efekcie prowadzą do dostarczenia klientowi produktu/usługi realizując tym samym cele biznesowe organizacji.
\section{Kryteria i klasyfikacje procesów biznesowych}
\hspace*{1.5 cm}W literaturze  można  spotkać  wiele  różnych  kryteriów  podziału  i klasyfikacji  procesów.  I  tak,  według  M.  Portera  [Porter,  1985,  s.  23]  wyróżnić  można  dwa  podstawowe  rodzaje  procesów:  \textbf{podstawowe}  i  \textbf{pomocnicze}. Do procesów podstawowych zaliczył on: 
\begin{itemize}
	\item Logistykę „na wejściu”, która obejmuje działania związane z przygotowaniem produkcji,
	\item wytwarzanie produktu, 
 	\item logistykę  „na  wyjściu”,  która obejmuje  działania związane ze sprzedażą,
	\item marketing,
	\item usługi posprzedażne.
\end{itemize}

\hspace*{1.5 cm}Natomiast wśród pomocniczych wymienił procesy związane z: 
\begin{itemize}
	\item Zarządzaniem całą jednostką,
	\item zarządzaniem zasobami ludzkimi,
	\item zaopatrzeniem,
	\item rozwojem mającym na celu doskonalenie produktów i procesów.
\end{itemize}

\hspace*{1.5 cm} Z  kolei  R.S.  Kaplan  i  R.  Cooper  [Kaplan,  Cooper,  2001, s.  99],  [Kaplan, Cooper, 2000, s. 200] wyróżnili procesy:
\begin{itemize}
	\item Innowacyjne – związane z określaniem rynku docelowego oraz two-rzeniem oferty produktu (usługi), 
	\item operacyjne  –  dotyczące  wytwarzania  produktu  (usługi)  i  dostarcza-nia go klientowi,
	\item obsługi posprzedażnej – obejmujące obsługę klienta po dostarczeniu mu produktu,
\end{itemize}
\hspace*{1.5 cm} Oraz podzielili je na:
\begin{itemize}
	\item Konieczne – których wykonanie jest niezbędne do dostarczenia war-tości  i  których  nie  można  obecnie  poprawić,  uprościć,  zredukować czy wyeliminować, 
	\item istotne – dostarczające wartość, aczkolwiek możliwe jest ich uprosz-czenie i poprawienie,
	\item nieistotne – które powinny być wyeliminowane. 
\end{itemize}
\hspace*{1.5 cm} Z kolei najobszerniejszą klasyfikację procesów przedstawiła organizacja APQC (American Productivity Quality Center), która przygotowała Model Klasyfikacji Procesów (z ang. Process Classification Framework PCF). Wspomniana  organizacja  zaproponowała  wprowadzenie  12  kategorii procesów, które podzielone zostały na dwie grupy:
\begin{itemize}
	\item Procesy  operacyjne  –  traktowane  jako  kluczowe,  dla  danego  podmiotu gospodarczego, 
	\item procesy  wspomagające  –  stanowiące  wsparcie  dla  procesów  operacyjnych.
\end{itemize}
\hspace*{1.5 cm} Do  pierwszej  grupy  zaliczono  5  kategorii  charakteryzujących  działalność  podmiotu  decydujących  o  zakresie  funkcjonowania przedsiębiorstwa,  natomiast  do  procesów  wspomagających  zakwalifikowano  7 kategorii,  które  przenikają  wszystkie  procesy  podstawowe  i  jednocześnie są podstawą do wydzielania ich na zewnątrz (outsourcing). Ilustrację  graficzną  klasyfikacji  procesów  zgodnej  z  modelem  APQC  przed-stawia tablica 1.

\begin{table}[h]
	\begin{center}
		\scalebox{1}{\begin{tabular}{ | l | l | }
				\hline
				Proces & Charakterystyka \\ \hline
				Operacyjne & 1.0 - Opracowanie wizji i strategii \\
						   & 2.0 - Rozwój i zarządzanie produktami i usługami  \\
						   & 3.0 -  Marketing i sprzedaż produktów i usług \\ 
						   & 4.0 - Zaopatrzenie, realizacja i dostawa produktów/usług \\
						   & 5.0 - Zarządzanie obsługą klienta		   \\ \hline
				Wspomagające & 6.0 -Organizacja i zarządzanie 	kapitałem ludzkim \\
				             & 7.0 - Zarządzanie technologią informatyczną\\
				             & 8.0 - Zarządzanie zasobami finansowymi\\
				             & 9.0 - Nabywanie, budowa i zarządzanie mieniem \\
				             & 10.0 - Zarządzanie ochroną środowiska oraz bezpieczeń-stwem i higieną pracy \\
				             & 11.0 - Zarządzanie relacjami zewnętrznymi\\
				             & 12.0 -  Zarządzanie wiedzą, doskonaleniem i zmianą\\
				\hline
		\end{tabular}}
	\end{center}
	\caption{\label{tab:bolts} Tablica 1. Klasyfikacja procesów według modelu APQC}
\end{table}
\hspace*{1.5 cm}Model APQC nie ogranicza się do przedstawienia listy 12 kategorii procesów,  rozwija  się  proponując  następującą  zależność:  kategoria  procesów  –  grupa  procesów  –  procesy  –  czynności  (działania).																																																				

\section{Przykłady biznes procesów w różnych rodzajach firm}
\subsection{Biznes procesy w firmie IT}
\subsubsection{Opis ogólny procesu}
\hspace*{1cm} Potrzebujemy zakupić nowe serwery z potrzeby zaspokojenia bieżącego zapotrzebowania na moc obliczeniową, przyrostu danych oraz wymiana obecnie posiadanego systemu na nowszy, bo jest brak kompatybilności z nowymi systemami operacyjnymi, brak wsparcia producenta, z tego wynika wysokie ryzyko awarii. Poniżej jest dokładne opisane problemy, informacje, wykresy oraz charakterystyki dotyczące obecnych i nowych serwerów.\newline
\subsubsection{Występujące problemy w firmie} 
		\begin{itemize}
		        \item \texttt{Na obecnym serwerze baza danych wzrasta się około 100 Gb co rok. Ze względu na duże przyrosty danych na obecnych serwerach brakuje miejsca. Brak przestrzeni na dane powoduje coraz wolniejsze działanie serwera. W przypadku awarii niemożliwym staje się szybka naprawa. W obecnej postój pracy wynosi 5 dni.}
	        	\item \texttt{Problemy z komunikacją oraz transferem danych między serwerami spowodowana przestarzałością sprzętu}
	        	\item \texttt{Niski poziom Uptime, poziom dostępności systemu określony w procentach. Informujący o czasie ciągłego i bezawaryjnego działania serwerów w ciągu roku. Można spodziewać się awarii w niedługim okresie.}
	        	\item \texttt{Okres gwarancji na sprzęt oraz oprogramowanie wygasł, wiąże się to z częstrzą możliwością występowania corac cięższych awarii.}
	        	\item \texttt{Przestarzały sprzęt wycofany już z produkcji i powszechnego użytku. Posiadany sprzęt zurzywa się i traci pierwotą niezawodność. Generuje to ryzyko braku możliwości wymiany części i podzespołów w przypadku uszkodzeń i awarii.}
	        	\item \texttt{ Stare oprogramowanie traci swoją kompatybilność z nowymi systemami i ma słabsze możliwości względem nowych.}
	        	\item \texttt{Niewystarczająca moc obliczeniowa powoduje powolną pracę serwerów. Dane przesyłane do aplikacji są coraz wolniej selekcjonowane oraz generowanie z długim czasem oczekiwania na wyniki. Wpływ na to ma szybkość procesowa, pamięć RAM, pojemność dysku, przepustowość, itp.}
		    \end{itemize}
	  \subsubsection{Wnioski} 
	
			 	\hspace*{1cm} Pozostaje mało miejsca i istnieje potrzeba rozszerzenia przestrzeni roboczej poprzez podłączenie dodatkowych urządzeń pamięci masowej lub zwiększenie pojemności istniejącej macierzy dyskowej. Niestety uniemożliwia to przestarzały sprzęt niekompatybilny z odpowiednikami dostępnymi na rynku.\newline
			\hspace*{1cm}  Wszystkie powyższe problemu powodują przerwy, awarie oraz  przestoje - \textbf{generuje to straty materialne oraz oszczerbki na reputacji firmy.}
			
			\subsubsection{Rozwiązanie problemów}
				\subsubsection{Obecny stan}
			  \includegraphics[scale=0.7]{obecny_stan}
			  	\subsubsection{Proponowany stan}
			  	\includegraphics[scale=0.7]{proponowany_stan} \newline
				\subsubsection{Klastry SQL Server 2019} 	
	\hspace*{1cm} Klastry to technologia, która pozwala łączyć kilka systemów komputerowych w całość. 
Cel klastra to z duplikowanie  serwerów i zapewnienie nieprzerwanego funkcjonowania aplikacji. W serwerach bazodanowych wszystkie informacje są zapamiętywane na każdej równolegle pracującej maszynie. W razie awarii systemy i urządzenia łączą się działającym serwerem, zawierającym te same dane.
\newline
	\hspace*{1cm}Klastry SQL Server 2019 to nowy sposób wykorzystania SQL Server do tworzenia wartościowych baz relacyjnych i przechowywania dużych zbiorów danych na jednej, skalowalnej platformie. Dzięki temu analiza danych oraz aplikacje łączące się z bazą są bardziej elastyczne i działają wydajniej. Duże klastry danych SQL Server 2019 zapewniają kompletną platformę i pomagają zwiększyć sukces  organizacji.
\subsubsection{Opis przepływu procesu biznesowego}






\subsection{Biznes procesy w firmach marketingowych}
\subsubsection{Wstep}
\hspace*{1 cm} W tym rozdziałe zostaną opisane procesy biznesowe zachodzące w firmach marketingowych. Zostaną również opisane przyczyny, dlaczego te procesy zachodzą i co one rozwiązują. Przykładowe procesy zachodzące w firmach marketingowych:
\begin{itemize}
	\item Opracowanie planu marketingowego,
	\item promocja firmy.
\end{itemize}
\subsubsection{Biznes proces o nazwie 'Opracowanie planu marketingowego'}
\paragraph{Ogólny opis procesu}\mbox{}\\
\hspace*{1 cm}Plan marketingowy firmy ma kluczowe znaczenie przy planowaniu działań, wraz z budżetem, planem produkcji i planem sprzedaży. Roczny plan przedsiębiorstwa, odpowiednio, określa ogólne cele przedsiębiorstwa, jednak plan marketingowy ma większe znaczenie nad innymi częściami ogólnego planu rocznego, ponieważ:
\begin{enumerate}
	\item Cele planu marketingowego mają bezpośredni wpływ na wyniki innych części planu rocznego,
	\item decyzje zapisane w planie marketingowym określają, co dokładnie wyprodukuje firma, po jakiej cenie i gdzie sprzedać, jak się reklamować.
\end{enumerate}
\hspace*{1 cm}Plan marketingowy służy jako kluczowy przewodnik dla pracy personalu zajmującego się działaniami związanymi z marketingiem w firmy.
\paragraph{Problemy które proces rozwiązuje}
\begin{enumerate}
	\item Firma rozwija się spontanicznie, od zwycięstwa do porażki,
	\item konflikty w programach rozwoju firmy,
	\item firma losowo kupuje produkty, stara się zdywersyfikować ofertę produktową w momencie, gdy wymagana jest koncentracja na głównej ofercie produktowej.
\end{enumerate} 
\newpage
\paragraph{Opis przepływu procesu biznesowego}\mbox{}\\
\hspace*{1 cm} Główne cele które firma chcę osiągnąć przy tworzeniu planu marketingowego następujące:
\begin{itemize}
	\item Systematyzacja, formalny opis pomysłów liderów firmy, przekazywanie ich pracownikom,
	\item ustalanie celów marketingowych, zapewniając kontrolę nad ich osiągnięciem,
	\item koncentracja i rozsądny podział zasobów firmy.
\end{itemize} 
\hspace*{1 cm} Proces składa się z sześciu kroków:
\begin{enumerate}
	\item Definicja misji przedsiębiorstwa,
	\item analiza SWOT,
	\item definiowanie celów i strategii organizacji jako całości,
	\item określenie zadań i programu działań dla ich realizacji,
	\item opracowanie planu marketingowego i monitorowanie jego realizacji,
	\item budżetowanie na potrzeby wdrożenia planu marketingowego.
\end{enumerate}
Więcej szczegółów na temat kroków:
\begin{enumerate}
	\item Na etapie definicji misji przedsiębiorstwa określa się cel wszystkich późniejszych obszarów w których firmy może działać.
	\item Analiza SWOT daje jasny obraz tego, gdzie znajduje się firma i co ona ma wewnątrz: analiza mocnych i słabych stron przedsiębiorstwa, a także szans i zagrożeń związanych z bezpośrednim otoczeniem przedsiębiorstwa (środowisko zewnętrzne);
	\item Trzecia sekcja stanowi podstawę do opracowania konkretnego programu działań marketingowych. Ten etap planu marketingowego obejmuje prognozowanie rozwoju rynków docelowych (segmentów), dynamikę procesów makro i mikroekonomicznych, a także możliwości zasobów przedsiębiorstwa. Na podstawie powyższej analizy wyznaczane główne cele  przedsiębiorstwa. Cele reprezentowane są w ustrukturyzowanym drzewie celów, w korzeniu którego znajduje się globalny cel firmy.
	\item Na czwartym etapie zadania działu marketingu są określane w ramach ogólnego planu przedsiębiorstwa i opracowywany jest program działań w celu rozwiązania tych problemów. Dla każdego docelowego segmentu rynku należy zaplanować odpowiednie towary (usługi) o wymaganej jakości i ilości, ich cenach, punktach sprzedaży i taktykach promocji dla konsumenta.
	\item Piąty krok pozwala uzyskać sam dokument, z określanymi wartościami parametrów, za pomocą których będzie monitorowana realizacja planu marketingowego.
	\item Budżet marketingowy-część planu marketingowego, która odzwierciedla planowane wartości przychodów, kosztów i zysków. Wysokość dochodu jest uzasadniona prognozowaną wielkością sprzedaży. Koszty są definiowane jako suma wszystkich rodzajów kosztów. Zatwierdzony budżet stanowi podstawę do zapewnienia produkcji towarów i działań marketingowych.
\end{enumerate}








\subsection{Zarządzanie procesami biznesowymi w sektorze hotelarskim}
\subsubsection{Wstep}
\hspace*{1 cm} W tym rodziałe zostaną opisane zarządzanie procesami biznesowymi w hotelach: z naciskiem na zapewnienie wysokiej jakości obsługi gości

Etapy pobytu gościa to: 
\begin{enumerate}
	\item przyjazd przed
	\item przyjazd i zakwaterowanie
	\item pobyt
	\item wyjazd
\end{enumerate} 
Każdy etap jest analizowany, a usprawnienia procesów biznesowych są stosowane w hotelu, który jest jednym z największych hoteli.\newline
\hspace*{1cm}Firmy zmuszone są do szybszego wprowadzania innowacji w swoich modelach biznesowych. Muszą skupić się na klientach, konkurencji i procesach. Te nowe modele biznesowe zostały opisane jako „system zarządzania procesami biznesowymi”.
\subsubsection{Procesy biznesowe w sektorze hotelarskim}
\hspace*{1cm}Działalność operacyjna hotelu obejmuje:
\begin{itemize}
	\item prognozowanie,
	\item podejmowanie decyzji, 
	\item kontrolę, 
	\item ocenę odpowiedzialności 
	\item ocenę całego procesu zarządzania. 
\end{itemize} 
Branża hotelarska jest prezentowana każdego dnia jako globalna branża z właścicielami i klientami na całym świecie. Korzystanie z usług hotelowych, takich jak: zakwaterowanie, restauracja, bar, centrum spa nie jest już uważane za luksus. Dzisiaj wszystkie te usługi są niezbędne wielu osobom w ich codziennym życiu.\newline
\subsubsection{Strategii uzyskania przewagi konkurencyjnej}
Istnieją pewne strategie mające na celu uzyskanie przewagi konkurencyjnej, a dwie główne to:
\begin{enumerate}
	\item stosowanie rabatów na ceny
	\item podnoszenie jakości usług w celu osiągnięcia lojalności klientów poprzez zapewnienie klientom wyjątkowych korzyści.
\end{enumerate}
Pierwsza alternatywa jest powszechnie stosowana w hotelach w celu zdobycia udziału w rynku. Z jednej strony może to najpierw pomóc w tym względzie, ale z drugiej strony istnieje również ryzyko negatywnego wpływu na średnio- i długoterminową rentowność hotelu.


Osiągnięcie wysokiego poziomu zadowolenia gości w hotelach jest niezbędne do wdrożenia systemu świadczenia usług wysokiej jakości. Proces ten składa się z 6 elementów, które można wykorzystać do wdrożenia systemu świadczenia usług wysokiej jakości. Ponadto powinni tworzyć nowe potrzeby i potrzeby dla swoich klientów.
\subsubsection{Faza rezerwacji}
\paragraph{Ogólny opis procesu}\mbox{}\\
Procedura rezerwacji różni się w zależności od hotelu i podlega systemowi rezerwacji stosowanemu przez hotel. Rezerwacja pokoju w hotelu może być:
\begin{enumerate}
	\item Bezpośredni kontakt gości (bezpośrednie spotkanie, telefon, e-mail itp.)
	\item Kontakty od przedstawicieli handlowych (globalny system dystrybucji, centralny system rezerwacji).
\end{enumerate}
\paragraph{Definicja problemu}\mbox{}\\
Po pierwsze, biuro rezerwacji w hotelu nie jest zarządzane przez Departament Front Office i działa tylko 8 godzin dziennie. Po drugie, Goście przedstawiają się w recepcji z prośbą o rezerwację pokoju, a następnie długo czekają, aż ich prośba zostanie rozpatrzona i ostatecznie przetworzona. Oczywiste jest, że współdziałanie dwóch oddzielnych działów (Departamentu Front Office i Departamentu Wykonawczego) jest niezbędne do zrealizowania pojedynczego wniosku o bezpośrednią rezerwację.
\paragraph{Sugestie dotyczące ulepszeń}\mbox{}\\
Biuro rezerwacji byłoby częścią departamentu front office zamiast działu wykonawczego.
\paragraph{Wyniki}
\begin{enumerate}
	\item Oszczędność czasu
	\item Komunikacja jest liniowa i bezpośrednia. Przepływ komunikacji odbywa się tylko w ramach jednego działu od początku do końca
	\item Jakość świadczonych usług jest wyższa: pracownicy (recepcjonistki) są uprawnieni do dokonywania rezerwacji i reagowania na prośby gości o rezerwację w czasie rzeczywistym (ponieważ biuro rezerwacji jest otwarte 8 ha dnia, podczas gdy dział recepcji jest czynny 24 ha dnia).
\end{enumerate}
\hspace{1cm} Powyższe elementy sprawiają, że hotel i dział obsługi klienta skutecznie wypełniają wniosek o rezerwację przez sprawny personel i zapewniają wysokiej jakości obsługę gości.
\subsubsection{Faza przybycia}
\paragraph{Ogólny opis procesu}\mbox{}\\
Hotel stosuje skomputeryzowaną procedurę odprawy, w której tylko goście wchodzący (goście bez wcześniejszej rezerwacji) muszą podać swoje dane osobowe i wypełnić formularz rejestracyjny, a goście dokonujący rezerwacji muszą potwierdzić swoje dane w systemie recepcji.\newline
\hspace*{1cm}Gdy gość zostanie umieszczony w pokoju, system front office automatycznie aktualizuje informacje o dostępności pokoju, a pokój zostanie zajęty w systemie. Dzięki metodom odprawy celem jest, aby operacje były jak najbardziej proste i wygodne. Ponadto ta metoda pomaga w zbieraniu prawdziwych informacji o gościu.\newline
\hspace*{1cm}W przypadku przekroczenia rezerwacji, gdy dwóch lub więcej gości przybywa do tego samego pokoju, recepcja może zaproponować jeden z nich lub zwiększyć skalę pokoju. Jeśli jednak hotel nie jest w stanie zorganizować dodatkowego gościa, należy dokonać specjalnego uzgodnienia dla gościa w innym hotelu o podobnej pozycji.
\paragraph{Definicja problemu.}\mbox{}\\
Po mapowaniu i analizie bieżących procesów odprawy w hotelu identyfikowane są następujące problemy:
\begin{enumerate}
	\item Nie wszyscy goście wypełniają formularz rejestracyjny podczas procesu odprawy. W takim przypadku faktyczny personel będzie miał czas na zarejestrowanie w systemie wszystkich danych wypełnionych przez gościa na karcie rejestracyjnej.
	\item Metoda płatności nie jest identyfikowana podczas procesu odprawy.
\end{enumerate}
\paragraph{Sugestie dotyczące ulepszeń}
\begin{enumerate}
	\item Korzystanie z karty rezerwacyjnej przez gościa podczas karty procesowej odprawy.
	\item Identyfikacja metody płatności i preautoryzacja karty kredytowej wraz z całkowitą szacunkową kwotą pieniędzy do zapłaty.
\end{enumerate}
\paragraph{Oczekiwane wyniki}
\begin{enumerate}
	\item Identyfikacja metody płatności i wykorzystanie autoryzacji karty kredyotewe podczas odprawy oznacza, że każdy gość zapewnił hotelowi koszty swojego pobytu.
	\item Karta rejestracyjna zapewni hotelowi bezpośrednie dane kontaktowe gości. Kontakty te będą bardzo cenne dla hotelu i mogą być wykorzystane w przyszłości do różnych kampanii marketingowych i ofert promocyjnych.
\end{enumerate}

\hspace{1cm} Optymalizacja wyżej wymienionych procedur przyczyni się do rozszerzenia bazy klientów i upewni się, że wszyscy przybywający goście ukończyli wstępną autoryzację karty kredytowej, aby uniknąć przypadków oszustwa.

\subsubsection{Pozostawanie w fazie zarządzania zaproszeniami gości i skargami}
\paragraph{Ogólny opis procesu}\mbox{}\\
\hspace{1cm} Podczas procesu obsługi gości dział front office powinien upewnić się, że goście są zadowoleni podczas pobytu. Kiedy goście mają jakieś specjalne wymagania, zwykle proszą o pomoc w recepcji. Może to dotyczyć: napraw pokoju, dodatkowych udogodnień w pokoju lub informacji. Jeśli te skargi lub prośby zostaną szybko rozwiązane, poprawia to zadowolenie gościa, a tym samym przyczynia się do świadczenia wysokiej jakości obsługi gości.
\paragraph{Definicja problemu.}\mbox{}\\
Po pierwsze, należy zauważyć, że w obu przypadkach (prośba i skarga) żaden hotel nie wypełnia żadnego raportu i nie przekazuje do odpowiedniego kierownika pod koniec zmiany na prośby i problemy napotkane podczas pracy. Po drugie, recepcjonista nie otrzymuje żadnych informacji zwrotnych od gospodyni domowej ani konserwacji, jeśli akcja została wykonana pomyślnie. Nie mając żadnych informacji, personel recepcji nie jest w stanie odpowiedzieć na wymagania gości.
\paragraph{Sugestie dotyczące ulepszeń}\mbox{}\\
Recepcjonista powinien przygotować raport z problemami pod koniec godzin pracy, w tym wnioski klientów i reklamacje, które napotkano podczas ich zmiany.

\paragraph{Oczekiwane wyniki}\mbox{}\\
Raport pomaga gromadzić dane, a następnie analizować je w celu rozwiązania problemów.Gdy raporty te są odpowiednio monitorowane, mogą przyczynić się do podjęcia odpowiednich środków w celu podniesienia jakości usług świadczonych gościowi.

\subsubsection{Wnioski}
Zastosowanie zarządzania procesami biznesowymi zapewnia wiele korzyści dla firm w usługach hotelowych. Istnieje wiele różnych procesów, takich jak dostarczanie niezbędnych danych wejściowych, proces obsługi klienta, sprzątanie i sprzątanie, proces jedzenia i napojów itp. Badanie to miało na celu usprawnienie obsługi biura obsługi przez biznes podejście do zarządzania procesami.

Kiedy operacje front office są analizowane w hotelu w trzech głównych fazach operacji, przed przybyciem, zameldowaniu i zakwaterowaniu, pobyt w fazach hotelowych wymaga poprawy ich usług.

\label{LastPage}~
\label{LastPageOfBackMatter}~		
\end{document}